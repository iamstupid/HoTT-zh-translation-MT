% 中文版宏定义
% Chinese Version Macros
% 注意:ctex 已由 ctexbook 文档类加载,此处不要重复加载

% 中文字体设置 (ctexbook 默认使用系统字体,如需自定义可取消注释)
% \setCJKmainfont{SimSun}[BoldFont=SimHei, ItalicFont=KaiTi]
% \setCJKsansfont{SimHei}
% \setCJKmonofont{FangSong}

% 章节标题中文化
% 注意:章节编号使用阿拉伯数字,章节标题显示使用中文格式
\renewcommand{\partname}{第}
\renewcommand{\thepart}{\chinese{part}}
\renewcommand{\chaptername}{第}
% \thechapter 保持阿拉伯数字,这样节编号为 "1.1" 而非 "一章.1"
% 章标题格式由 ctexbook 的 chapter/name 和 chapter/number 控制

% ctex 格式设置
% 章标题显示格式
\ctexset{
  chapter = {
    name = {第,章},
    number = \arabic{chapter}
  },
  part = {
    name = {第,部分},
    number = \arabic{part}
  }
}

% 强制重置 \thechapter 和 \thepart 为纯阿拉伯数字
% 这是为了确保 cleveref 正常工作
\makeatletter
\renewcommand{\thechapter}{\@arabic\c@chapter}
\renewcommand{\thepart}{\@arabic\c@part}
\makeatother
\renewcommand{\contentsname}{目录}
\renewcommand{\listfigurename}{插图目录}
\renewcommand{\listtablename}{表格目录}
\renewcommand{\bibname}{参考文献}
\renewcommand{\indexname}{索引}
\renewcommand{\figurename}{图}
\renewcommand{\tablename}{表}
\renewcommand{\appendixname}{附录}

% 定理环境中文化 (使用 \providecommand 避免重复定义错误)
\providecommand{\theoremname}{定理}
\providecommand{\lemmaname}{引理}
\providecommand{\corollaryname}{推论}
\providecommand{\propositionname}{命题}
\providecommand{\definitionname}{定义}
\providecommand{\remarkname}{注记}
\providecommand{\examplename}{例}
\providecommand{\exercisename}{练习}
\renewcommand{\proofname}{证明}  % 已由 amsthm 定义,需用 renewcommand
\providecommand{\solutionname}{解}
\providecommand{\notename}{注}
\providecommand{\axiomname}{公理}

% HoTT 特定术语命令
\newcommand{\zhtype}{类型}
\newcommand{\zhuniverse}{宇宙}
\newcommand{\zhpath}{路径}
\newcommand{\zhhomotopy}{同伦}
\newcommand{\zhequivalence}{等价}
\newcommand{\zhunivalence}{泛等性}
\newcommand{\zhcontractible}{可缩的}
\newcommand{\zhtruncation}{截断}
\newcommand{\zhfiber}{纤维}
\newcommand{\zhfibration}{纤维化}
\newcommand{\zhinduction}{归纳}
\newcommand{\zhrecursion}{递归}
\newcommand{\zhtransport}{传输}

% 中英文对照命令
% 用于在正文中标注英文原词
% 例如: \en{homotopy type theory}{同伦类型论}
\newcommand{\en}[2]{#2\footnote{英文: #1}}

% 术语首次出现时使用
% 例如: \term{同伦类型论}{homotopy type theory}
\newcommand{\term}[2]{\textbf{#1} (#2)}

% 数学术语(不需要翻译的专有名词)
\newcommand{\mathterm}[1]{\textit{#1}}

% narrowmultline 环境 (来自原书的 opt-*.tex)
\newenvironment{narrowmultline*}{\csname multline*\endcsname}{\csname endmultline*\endcsname}
\newenvironment{narrowmultline}{\csname multline\endcsname}{\csname endmultline\endcsname}

% 断行命令 (用于 narrowmultline 环境)
\providecommand{\narrowbreak}{\\}

% narrowamp 命令 (用于对齐环境中的条件换行)
\providecommand{\narrowamp}{}

% 来自原书 opt-*.tex 的布局命令 (作为空操作)
\providecommand{\OPTwidow}{}
\providecommand{\OPTsmalltable}{}
\providecommand{\mentalpause}{\par\medskip}

% narrowequation 环境 (用于窄方程)
\providecommand{\narrowequation}[1]{\begin{equation*}#1\end{equation*}}

% 修复 \eqref 方程引用问题
% hyperref 和 cleveref 可能会干扰 amsmath 的 \eqref
% 重新定义以确保正常工作
\makeatletter
\renewcommand{\eqref}[1]{\textup{(\ref{#1})}}
\makeatother

% cleveref 中文配置
% 首先设置引用名称 (用于默认格式和多引用)
\crefname{chapter}{章}{章}
\Crefname{chapter}{章}{章}
\crefname{section}{节}{节}
\Crefname{section}{节}{节}
\crefname{part}{部分}{部分}
\Crefname{part}{部分}{部分}
\crefname{appendix}{附录}{附录}
\Crefname{appendix}{附录}{附录}
\crefname{figure}{图}{图}
\Crefname{figure}{图}{图}
\crefname{table}{表}{表}
\Crefname{table}{表}{表}
\crefname{equation}{公式}{公式}
\Crefname{equation}{公式}{公式}

% 章的格式 - 使用阿拉伯数字
\crefformat{chapter}{第~#2#1#3~章}
\Crefformat{chapter}{第~#2#1#3~章}
\crefmultiformat{chapter}{第~#2#1#3~章}{ 和第~#2#1#3~章}{、第~#2#1#3~章}{ 和第~#2#1#3~章}
\Crefmultiformat{chapter}{第~#2#1#3~章}{ 和第~#2#1#3~章}{、第~#2#1#3~章}{ 和第~#2#1#3~章}
\crefrangeformat{chapter}{第~#3#1#4~章至第~#5#2#6~章}
\Crefrangeformat{chapter}{第~#3#1#4~章至第~#5#2#6~章}

% 节的格式
\crefformat{section}{第~#2#1#3~节}
\Crefformat{section}{第~#2#1#3~节}
\crefmultiformat{section}{第~#2#1#3~节}{ 和第~#2#1#3~节}{、第~#2#1#3~节}{ 和第~#2#1#3~节}
\Crefmultiformat{section}{第~#2#1#3~节}{ 和第~#2#1#3~节}{、第~#2#1#3~节}{ 和第~#2#1#3~节}
\crefrangeformat{section}{第~#3#1#4~节至第~#5#2#6~节}
\Crefrangeformat{section}{第~#3#1#4~节至第~#5#2#6~节}

% 小节
\crefformat{subsection}{第~#2#1#3~小节}
\Crefformat{subsection}{第~#2#1#3~小节}

% 部分
\crefformat{part}{第#2#1#3部分}
\Crefformat{part}{第#2#1#3部分}
\crefmultiformat{part}{第#2#1#3部分}{ 和第#2#1#3部分}{、第#2#1#3部分}{ 和第#2#1#3部分}
\Crefmultiformat{part}{第#2#1#3部分}{ 和第#2#1#3部分}{、第#2#1#3部分}{ 和第#2#1#3部分}

% 附录
\crefformat{appendix}{附录~#2#1#3}
\Crefformat{appendix}{附录~#2#1#3}

% 图表公式
\crefformat{figure}{图~#2#1#3}
\Crefformat{figure}{图~#2#1#3}
\crefformat{table}{表~#2#1#3}
\Crefformat{table}{表~#2#1#3}
\crefformat{equation}{公式~(#2#1#3)}
\Crefformat{equation}{公式~(#2#1#3)}

% 定理类型的引用名称
\crefname{thm}{定理}{定理}
\Crefname{thm}{定理}{定理}
\crefname{lem}{引理}{引理}
\Crefname{lem}{引理}{引理}
\crefname{cor}{推论}{推论}
\Crefname{cor}{推论}{推论}
\crefname{defn}{定义}{定义}
\Crefname{defn}{定义}{定义}
\crefname{rmk}{注记}{注记}
\Crefname{rmk}{注记}{注记}
\crefname{eg}{例}{例}
\Crefname{eg}{例}{例}
\crefname{ex}{练习}{练习}
\Crefname{ex}{练习}{练习}

% 定理类型的引用格式
\crefformat{thm}{定理~#2#1#3}
\Crefformat{thm}{定理~#2#1#3}
\crefformat{lem}{引理~#2#1#3}
\Crefformat{lem}{引理~#2#1#3}
\crefformat{cor}{推论~#2#1#3}
\Crefformat{cor}{推论~#2#1#3}
\crefformat{defn}{定义~#2#1#3}
\Crefformat{defn}{定义~#2#1#3}
\crefformat{rmk}{注记~#2#1#3}
\Crefformat{rmk}{注记~#2#1#3}
\crefformat{eg}{例~#2#1#3}
\Crefformat{eg}{例~#2#1#3}
\crefformat{ex}{练习~#2#1#3}
\Crefformat{ex}{练习~#2#1#3}

% 修复 ctex 与 cleveref 章节引用的兼容性问题
% ctex 的计数器格式与 cleveref 不完全兼容
% 创建自定义章节引用命令来绕过这个问题

% 自定义章引用命令 - 直接输出 "第X章" 格式
\newcommand{\zhchapter}[1]{%
  第\ref{#1}章%
}

% 自定义节引用命令
\newcommand{\zhsection}[1]{%
  第~\ref{#1}~节%
}

% 自定义部分引用命令
\newcommand{\zhpart}[1]{%
  第\ref{#1}部分%
}

% 保存原始的 cref 命令并重新定义
% 由于 ctex 污染了 cleveref 的内部数据,我们根据标签前缀来判断类型
% 并直接使用 \ref 来避免 cleveref 的损坏数据
\makeatletter
\let\origcref\cref
\let\origCref\Cref

% 单标签引用的内部命令
\newcommand{\@singlecref}[1]{%
  \@ifundefined{r@#1}{??}{%
    \in@{cha:}{#1}%
    \ifin@
      第~\ref{#1}~章%
    \else
      \in@{part:}{#1}%
      \ifin@
        第\ref{#1}部分%
      \else
        \in@{sec:}{#1}%
        \ifin@
          第~\ref{#1}~节%
        \else
          \in@{thm:}{#1}%
          \ifin@
            定理~\ref{#1}%
          \else
            \in@{lem:}{#1}%
            \ifin@
              引理~\ref{#1}%
            \else
              \in@{cor:}{#1}%
              \ifin@
                推论~\ref{#1}%
              \else
                \in@{defn:}{#1}%
                \ifin@
                  定义~\ref{#1}%
                \else
                  \in@{rmk:}{#1}%
                  \ifin@
                    注记~\ref{#1}%
                  \else
                    \in@{eg:}{#1}%
                    \ifin@
                      例~\ref{#1}%
                    \else
                      \in@{ex:}{#1}%
                      \ifin@
                        练习~\ref{#1}%
                      \else
                        \in@{eq:}{#1}%
                        \ifin@
                          公式~(\ref{#1})%
                        \else
                          \in@{fig:}{#1}%
                          \ifin@
                            图~\ref{#1}%
                          \else
                            \in@{tab:}{#1}%
                            \ifin@
                              表~\ref{#1}%
                            \else
                              % 其他类型直接使用 ref 避免 cleveref 损坏数据
                              \ref{#1}%
                            \fi
                          \fi
                        \fi
                      \fi
                    \fi
                  \fi
                \fi
              \fi
            \fi
          \fi
        \fi
      \fi
    \fi
  }%
}

% 处理逗号分隔的多标签引用
\newcounter{@crefcount}
\newcommand{\@processcreflist}[1]{%
  \setcounter{@crefcount}{0}%
  \@for\@crefitem:=#1\do{%
    \stepcounter{@crefcount}%
    \ifnum\value{@crefcount}>1 { 和 }\fi
    \expandafter\@singlecref\expandafter{\@crefitem}%
  }%
}

% 重新定义 cref,支持逗号分隔的多标签
\renewcommand{\cref}[1]{%
  \in@{,}{#1}%
  \ifin@
    \@processcreflist{#1}%
  \else
    \@singlecref{#1}%
  \fi
}

\renewcommand{\Cref}[1]{\cref{#1}}

% 处理 crefrange - 简化为 "X 至 Y" 格式
\let\origcrefrange\crefrange
\renewcommand{\crefrange}[2]{%
  \@singlecref{#1}~至~\@singlecref{#2}%
}
\makeatother
