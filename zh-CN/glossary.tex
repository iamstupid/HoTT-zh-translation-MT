% 同伦类型论术语表 / HoTT Glossary
% 本文件定义了同伦类型论中的标准术语翻译
%
% 主要参考来源:
% - 维基百科中文版 (zh.wikipedia.org)
% - 香蕉空间 (bananaspace.org)
% - 知乎 HoTT 相关文章

% ============================================
% 核心概念 / Core Concepts
% ============================================

% Homotopy Type Theory (HoTT)   同伦类型论
% Univalent Foundations         泛等基础
% Univalence Axiom              泛等公理
% Univalence                    泛等性
% Type Theory                   类型论
% Dependent Type Theory         依赖类型论 / 依值类型论
% Martin-Löf Type Theory        Martin-Löf 类型论

% ============================================
% 类型 / Types
% ============================================

% type                          类型
% universe                      宇宙
% universe level                宇宙层级
% dependent type                依赖类型 / 依值类型
% dependent function type       依赖函数类型 (Π-类型)
% dependent pair type           依赖对类型 (Σ-类型)
% function type                 函数类型
% product type                  积类型
% sum type / coproduct type     和类型 / 余积类型
% identity type                 相等类型 / 恒等类型
% path type                     路径类型
% higher inductive type (HIT)   高阶归纳类型
% inductive type                归纳类型
% coinductive type              余归纳类型
% W-type                        W-类型
% record type                   记录类型
% sigma type (Σ-type)           Σ-类型 / 依赖和类型
% pi type (Π-type)              Π-类型 / 依赖积类型
% empty type                    空类型
% unit type                     单位类型
% boolean type                  布尔类型
% natural number type           自然数类型
% truncation                    截断
% propositional truncation      命题截断 / 逻辑截断
% interval type                 区间类型

% ============================================
% 同伦论术语 / Homotopy Theory Terms
% ============================================

% path                          路径 / 道路
% loop                          回路
% homotopy                      同伦
% homotopy equivalence          同伦等价
% contractible                  可缩的
% fiber                         纤维
% fibration                     纤维化
% n-type                        n-类型
% set (0-type)                  集合 (0-类型)
% mere proposition (-1-type)    纯命题 (-1-类型)
% groupoid                      广群
% ∞-groupoid                    ∞-广群 / 无穷广群
% higher groupoid               高阶广群
% fundamental group             基本群
% loop space                    回路空间
% suspension                    纬悬 / 悬垂
% pushout                       推出
% pullback                      拉回
% circle (S¹)                   圆 / 圆周 (S¹)
% sphere (Sⁿ)                   球面 (Sⁿ)
% torus                         环面
% truncation level              截断层级
% connected                     连通的
% n-connected                   n-连通的
% pointed type                  有基点类型
% base point                    基点

% ============================================
% 逻辑与命题 / Logic and Propositions
% ============================================

% proposition                   命题
% proof                         证明
% witness                       证人 / 见证
% evidence                      证据
% axiom                         公理
% theorem                       定理
% lemma                         引理
% corollary                     推论
% definition                    定义
% remark                        注记
% example                       例
% exercise                      练习
% propositional equality        命题相等 / 命题等同
% judgmental equality           判断相等 / 依判断相等
% definitional equality         定义相等
% decidable                     可判定的
% decidable equality            可判定相等
% proof irrelevance             证明无关性
% excluded middle               排中律
% axiom of choice               选择公理
% law of double negation        双重否定律
% propositional extensionality  命题外延性
% function extensionality       函数外延性
% judgment                      论断 / 判断
% context                       语境 / 上下文
% assumption                    前提
% hypothesis                    假设

% ============================================
% 等价与同构 / Equivalences and Isomorphisms
% ============================================

% equivalence                   等价
% quasi-inverse                 拟逆
% bi-invertible map             双可逆映射
% half-adjoint equivalence      半伴随等价
% isomorphism                   同构
% bijection                     双射
% embedding                     嵌入
% surjection                    满射
% injection                     单射
% retraction                    收缩
% section                       截面

% ============================================
% 范畴论 / Category Theory
% ============================================

% category                      范畴
% functor                       函子
% natural transformation        自然变换
% adjunction                    伴随
% adjoint functor               伴随函子
% limit                         极限
% colimit                       余极限
% initial object                始对象
% terminal object               终对象
% morphism                      态射
% object                        对象
% precategory                   预范畴
% univalent category            泛等范畴
% strict category               严格范畴
% wild category                 狂野范畴
% Rezk completion               Rezk 完备化

% ============================================
% 集合论 / Set Theory
% ============================================

% set                           集合
% subset                        子集
% power set                     幂集
% cardinal                      基数
% ordinal                       序数
% well-ordering                 良序
% Zorn's lemma                  佐恩引理
% cumulative hierarchy          累积层级

% ============================================
% 实数与分析 / Real Numbers and Analysis
% ============================================

% real number                   实数
% Cauchy real                   柯西实数
% Dedekind cut                  戴德金分割
% rational number               有理数
% integer                       整数
% supremum                      上确界
% infimum                       下确界
% limit                         极限
% continuity                    连续性
% convergence                   收敛

% ============================================
% 证明助手 / Proof Assistants
% ============================================

% Coq                           Coq
% Agda                          Agda
% Lean                          Lean
% proof assistant               证明助手
% formalization                 形式化
% mechanized proof              机械化证明

% ============================================
% 类型论操作 / Type-Theoretic Operations
% ============================================

% computation rule              计算规则
% uniqueness principle          唯一性原理
% elimination rule              消去规则
% introduction rule             引入规则
% induction principle           归纳原理
% recursion principle           递归原理
% transport                     传输
% ap (action on paths)          路径作用 (ap)
% apd (dependent ap)            依赖路径作用 (apd)
% encode-decode method          编码-解码方法
% fundamental theorem           基本定理
% J rule / path induction       J 规则 / 路径归纳

% ============================================
% 立方类型论 / Cubical Type Theory
% ============================================

% cubical type theory           立方类型论
% interval                      区间
% face                          面
% composition                   复合
% Kan operation                 Kan 操作
% glueing                       粘合

% ============================================
% 术语对照表 (按英文字母排序)
% Terminology Cross-Reference (Alphabetical)
% ============================================
%
% A
% action on paths (ap)          路径作用
% adjoint functor               伴随函子
% adjunction                    伴随
% axiom                         公理
% axiom of choice               选择公理
%
% B
% base point                    基点
% bi-invertible map             双可逆映射
% bijection                     双射
%
% C
% category                      范畴
% circle                        圆 / 圆周
% colimit                       余极限
% computation rule              计算规则
% connected                     连通的
% context                       语境
% contractible                  可缩的
% coproduct type                余积类型
%
% D
% decidable                     可判定的
% Dedekind cut                  戴德金分割
% definitional equality         定义相等
% dependent type                依赖类型
%
% E
% elimination rule              消去规则
% embedding                     嵌入
% empty type                    空类型
% encode-decode method          编码-解码方法
% equivalence                   等价
% excluded middle               排中律
%
% F
% fiber                         纤维
% fibration                     纤维化
% function extensionality       函数外延性
% function type                 函数类型
% functor                       函子
% fundamental group             基本群
%
% G
% groupoid                      广群
%
% H
% half-adjoint equivalence      半伴随等价
% higher inductive type         高阶归纳类型
% homotopy                      同伦
% homotopy equivalence          同伦等价
% Homotopy Type Theory          同伦类型论
%
% I
% identity type                 相等类型
% induction principle           归纳原理
% inductive type                归纳类型
% initial object                始对象
% injection                     单射
% interval                      区间
% introduction rule             引入规则
% isomorphism                   同构
%
% J
% judgment                      论断
% judgmental equality           判断相等
%
% L
% lemma                         引理
% limit                         极限
% loop                          回路
% loop space                    回路空间
%
% M
% mere proposition              纯命题
% morphism                      态射
%
% N
% n-connected                   n-连通的
% n-type                        n-类型
% natural number type           自然数类型
% natural transformation        自然变换
%
% O
% object                        对象
%
% P
% path                          路径
% path induction                路径归纳
% path type                     路径类型
% pi type (Π-type)              Π-类型
% pointed type                  有基点类型
% precategory                   预范畴
% product type                  积类型
% proof                         证明
% proof assistant               证明助手
% proof irrelevance             证明无关性
% proposition                   命题
% propositional equality        命题相等
% propositional extensionality  命题外延性
% propositional truncation      命题截断
% pullback                      拉回
% pushout                       推出
%
% Q
% quasi-inverse                 拟逆
%
% R
% real number                   实数
% recursion principle           递归原理
% retraction                    收缩
%
% S
% section                       截面
% set                           集合
% sigma type (Σ-type)           Σ-类型
% sphere                        球面
% strict category               严格范畴
% sum type                      和类型
% supremum                      上确界
% surjection                    满射
% suspension                    纬悬
%
% T
% terminal object               终对象
% theorem                       定理
% torus                         环面
% transport                     传输
% truncation                    截断
% truncation level              截断层级
% type                          类型
% Type Theory                   类型论
%
% U
% uniqueness principle          唯一性原理
% unit type                     单位类型
% Univalence Axiom              泛等公理
% Univalent Foundations         泛等基础
% univalent category            泛等范畴
% universe                      宇宙
% universe level                宇宙层级
%
% W
% W-type                        W-类型
% well-ordering                 良序
% wild category                 狂野范畴
% witness                       证人
